\documentclass{amsart}
%\usepackage[dvips]{graphicx}
\usepackage{graphicx}
\graphicspath{ {.} }
\setkeys{Gin}{width=0.8\linewidth,keepaspectratio=true}
%\usepackage[latin1]{inputenc}
\usepackage{natbib}
\bibpunct{(}{)}{;}{a}{,}{;}
\usepackage{color}
\usepackage{url}
\usepackage{epsfig}
\usepackage{algorithm}
\usepackage{algorithmic}
\usepackage[algo2e]{algorithm2e}
\usepackage{longtable}
\newcommand{\elite}{\mathcal{E}}
\usepackage{rotating}
\usepackage{subfigure}
\usepackage{lscape}
%\usepackage{subcaption}
\usepackage{multirow}
\usepackage{adjustbox}
\usepackage{comment}
\usepackage{setspace}
\doublespace



\begin{document}

%%\title[Amazon Locker capacity management]{Amazon Locker capacity management:\\
%%Bridging the gap between machine learning
%%\\and operations research}
\title[Aplicação do GRASP e BRKGA]{Aplicação do GRASP e BRKGA no recobrimento Máximo Ponderado}


\author[V. Lucas]{Lucas}
\address[V. Lucas]{ITA, São José dos Campos, SP}
\email[V. Lucas]{aaaa@ita.br}

\author[P. Gerson]{Gerson}
\address[P. Gerson]{ITA, São José dos Campos, SP}
\email[P. Gerson]{gersonprimo@gmail.com}


\begin{abstract}

Como parte pratica do curso `Ciencia de dados com Metaholisticas` foi feito a simulação de

\texttt{pdflatex template.tex; bibtex template.aux; pdflatex template.tex}.
\end{abstract}
%
%\date{November 17, 2022}
%\thanks{ITA}
\maketitle


\section{Apresentação (Problemas de otimização e heurísticas)}
\label{s_sec1}

Heurísticas têm sido aplicadas para resolução de diversos problemas de otimização [1]. Um problema de otimização pode ser descrito através de uma formulação matemática onde temos uma função real que pretendemos maximizar ou minimizar com um determinado número de restrições associadas [2]. Para resolver esse tipo de problema, diversos mecanismos podem ser empregados, desde métodos de resolução exata algébricos e numéricos. Contudo, esses métodos ou são limitados (como alguns métodos algébricos) ou podem explorar um número tão grande de soluções até encontrar a solução ótima, que na prática eles se tornam inviáveis de serem aplicados a problemas maiores, que em geral, são problemas que aparecem no dia-a-dia [3]. Dadas essas restrições, uma alternativa que vem se mostrando adequada para problemas de tal natureza são as heurísticas. Heurísticas consistem de métodos construídos de forma eficiente baseados em propriedades estruturais de um problema em particular, que de forma geral, oferecem boas soluções para aquele problema, porém sem garantia de otimalidade (encontrar a melhor solução para todos os casos de um problema em particular) [4].

\section{Segunda Seção}
\label{s_sec2}

Aqui está o meio. Bla, Bla, bla.  \citet{Pla99a} disse isso.
Isso foi dito em \citet{Pla99a}.


\begin{comment}
\begin{table}[ht]
\centering
\caption{Um exemplo de tabela}
\label{t_simulation}
\begin{tabular}{|c|c|}
\hline
Locker Name & $\%$ Accuracy \\ \hline
\hline
Boson                  & 98.9        \\
Seth                   & 92.6        \\
Berlin                 & 96.1        \\
Grape                & 99.3 	\\
\hline
\end{tabular}
\end{table}
\end{comment}

\section{Última Seção}
\label{s_conclusion}

Bla, bla, bla.
Bla, bla, bla.
Bla, bla, bla.
Bla, bla, bla.
Bla, bla, bla.
Bla, bla, bla.
Bla, bla, bla.
Bla, bla, bla.
Bla, bla, bla.
Bla, bla, bla.
Bla, bla, bla.
Bla, bla, bla.
Bla, bla, bla.
Bla, bla, bla.
Bla, bla, bla.
Bla, bla, bla.
Bla, bla, bla.
Bla, bla, bla.
Bla, bla, bla.
Bla, bla, bla.
Bla, bla, bla.
Bla, bla, bla.
Bla, bla, bla.
Bla, bla, bla.
Bla, bla, bla.
Bla, bla, bla.
Bla, bla, bla.
Bla, bla, bla.
Bla, bla, bla.
Bla, bla, bla.
Bla, bla, bla.
Bla, bla, bla.
Bla, bla, bla.
Bla, bla, bla.
Bla, bla, bla.
Bla, bla, bla.
Aqui termina tudo.

\bibliographystyle{plainnat}
\bibliography{references}

\end{document}
